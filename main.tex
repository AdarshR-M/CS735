\documentclass{beamer}
\usepackage{hyperref}
\usepackage{physics}
\usepackage{amsmath}
\usepackage{tikz}
\usepackage{mathdots}
\usepackage{yhmath}
\usepackage{cancel}
\usepackage{color}
\usepackage{siunitx}
\usepackage{array}
\usepackage{multirow}
\usepackage{amssymb}
\usepackage{gensymb}
\usepackage{tabularx}
\usepackage{extarrows}
\usepackage{booktabs}
\usetikzlibrary{fadings}
\usetikzlibrary{patterns}
\usetikzlibrary{shadows.blur}
\usetikzlibrary{shapes}
\usetheme{Madrid}
% \usecolortheme{beaver}
\definecolor{myblue}{RGB}{10, 35, 68}
\definecolor{mygreen}{RGB}{42, 96, 91}

\begin{filecontents}{beamercolorthememyblue.sty}
\mode<presentation>
\setbeamercolor{section in toc}{fg=black,bg=white}
\setbeamercolor{alerted text}{fg=myblue!80!gray}
\setbeamercolor*{palette primary}{bg=myblue!95,fg=white}
\setbeamercolor*{palette secondary}{fg=myblue!70!black,bg=gray!15!white}
\setbeamercolor*{palette tertiary}{bg=myblue!80!black,fg=gray!10!white}
\setbeamercolor*{palette quaternary}{fg=myblue,bg=gray!5!white}

\setbeamercolor*{sidebar}{fg=myblue,bg=gray!15!white}

\setbeamercolor*{palette sidebar primary}{fg=myblue!10!black}
\setbeamercolor*{palette sidebar secondary}{fg=white}
\setbeamercolor*{palette sidebar tertiary}{fg=myblue!50!black}
\setbeamercolor*{palette sidebar quaternary}{fg=gray!10!white}

%\setbeamercolor*{titlelike}{parent=palette primary}
\setbeamercolor{titlelike}{parent=palette primary}
\setbeamercolor{frametitle}{parent=palette primary}
\setbeamercolor{frametitle right}{bg=gray!60!white}

\setbeamercolor*{separation line}{}
\setbeamercolor*{fine separation line}{}
\mode
<all>
\end{filecontents}

\renewcommand{\r}[1]{\text{\rmfamily{\textbf{#1}}}}
\renewcommand{\b}[1]{\boldsymbol{#1}}
\newcommand{\zero}[0]{\b{0}}
\newcommand{\cmd}[0]{\text{\rmfamily{\textbf{cmd}}}}
\newcommand{\test}[0]{\text{\rmfamily{\textbf{test}}}}
\newcommand{\inc}[0]{\text{\rmfamily{\textbf{inc}}}}
\newcommand{\dec}[0]{\text{\rmfamily{\textbf{dec}}}}
\newcommand{\lo}[0]{\text{\rmfamily{\textbf{loop}}}}
\newcommand{\loopam}[2]{\text{\rmfamily{\textbf{loop} at most $#1$ times $#2$}}}
\newcommand{\Or}[0]{\text{\rmfamily{\textbf{or}}}}
\newcommand{\prog}[1]{\xrightarrow[]{#1}}

\usecolortheme{myblue}

% \colorlet{beamer@blendedblue}{myblue}
\definecolor{links}{HTML}{2A1B81}
\hypersetup{colorlinks,linkcolor=,urlcolor=links}

\title[Reachability in Petri Nets]{The Reachability Problem for Petri Nets is Not Primitive Recursive}
\author[CS 735]{CS 735 \\ Based on paper by J\'{e}r\^{o}me Leroux}
% \institute{Adarsh}
\date{April 2, 2024}

\AtBeginSection[]
{
    \begin{frame}
        \frametitle{Table of Contents}
        \tableofcontents[currentsection]
    \end{frame}
}

\begin{document}

\frame{\titlepage}

% \begin{frame}
% \frametitle{Table of Contents}
% \tableofcontents[currentsection]
% \end{frame}

\section{Introduction}

\begin{frame}{Introduction}
    \begin{itemize}
         \item First we introduce general, test-free and checking programs.
        \item Connection between petri-nets reachability and reachability in checking programs. 
        \pause
        \item We show that reachability in a $(2n+4)$-dimensional checking program is in $\mathcal{O}(F_{n+1}(n)) = F_\omega(n)$, hence, the problem is $\mathbb{F}_\omega$-hard. 
        \item J. Leroux and S. Schmitz [LICS 2019] shows that the problem is in $\mathbb{F}_\omega$.
        \item Hence we conclude that the reachabiliy in checking programs, and, by extension petri-nets, is $\mathbb{F}_\omega$-complete.  
    \end{itemize}
   
\end{frame}

\section{Programs}

\begin{frame}{Counters and Configurations}

\begin{itemize}
    \item Consider an infinite, countable set of \textbf{counters} $\mathbb{C}$ (just like we have a countable set of variables in propositional logic).

    \item A \textbf{configuration} $\rho$ is a map from $\mathbb{C}$ to $\mathbb{N}$ such that the number of non-zero counters is finite. That is,
    $$|\{c \, | \, \rho(c) > 0,  c\in \mathbb{C}\}| < \infty$$
    \item $\zero{}$ denotes the configuration where $\rho(c)=0$ $\forall$ $c\in \mathbb{C}$.
\end{itemize}
    
\end{frame}

\begin{frame}{Command}
A relation $\prog{\cmd}$ is defined by $\alpha \prog{\cmd} \beta$ over configurations as 

\begin{itemize}
    \item If \cmd{} is \inc(c), $\beta(c) = \alpha(c)+1$ and $\forall$ $x\neq c$, $\beta(x) = \alpha(x)$. Informally, this is equivalent to $c$ += 1. 
    \pause
    \item If \cmd{} is \dec(c), $\beta(c) = \alpha(c)-1$ and $\forall$ $x\neq c$, $\beta(x) = \alpha(x)$. Informally, this is equivalent to $c$ $-$= 1. 
    \pause
    \item If \cmd{} is \test(c), $\beta(c) = \alpha(c) = 0$ and $\forall$ $x\neq c$, $\beta(x) = \alpha(x)$. Informally, this is equivalent to $c$ == 0.
\end{itemize}

Also $\r{r}(c_1,\dots,c_n) = \r{r}(c_1);\dots;\r{r}(c_n)$, $\r{r} \in \{\inc,\dec,\test\}$.
\end{frame}

\begin{frame}{Programs}
\begin{itemize}
    \item A \textbf{general program} (or program) $M$ is 
    $$M := \cmd \,|\, \lo\, M_0 |\, M_1 ; M_2 \,|\, M_1 \,\Or \,M_2$$
    where $M_0$, $M_1$ and $M_2$ are programs.
    \pause
    \item Every program $M$ is associated with the binary relation $\prog{M}$,
    $$\prog{M} \quad = \quad \begin{cases}
        \prog{\cmd} & \text{if } M = \cmd \\
        \big(\prog{M_0}\big)^* & \text{if } M = \lo\\
        \prog{M_1};\prog{M_2} & \text{if } M = M_1;M_2\\
        \prog{M_1}\cup\prog{M_2} & \text{if } M = M_1 \, \Or \, M_2\\
    \end{cases}$$
    \item To make the notation less cluttered, $M^{(n)} = M;\dots;M\, (n \text{ times})$.
\end{itemize}
\end{frame}


\begin{frame}{Program : Examples}
\begin{itemize}
    \item Consider the following program:
$$M = \left\vert \begin{array}{l}
     \lo \\
     \quad \inc(c_1,c_4);\dec(c_2) \\
    \dec(c_3)^{(3)} \;\Or\; \inc(c_1) \\
\end{array} \right. $$
\pause
The only reachable configuration of $\zero$ is $(1,0,0)$. That is,
$$\zero \prog{M} \rho $$
iff $\rho(c_1) = 1$ and $\rho(x) = 0$ $\forall$ $x \in \mathbb{C}\setminus \{c_1\}$.
\pause
\item Consider another program $M' = M;\test(c_1,c_3)$. There exists \textbf{no} configuration $\rho$ such that $\zero \prog{M'} \rho$.
\end{itemize}



\end{frame}


\begin{frame}{Test Free and Checking Programs}
    \begin{itemize}
        \item A program is said to be \textbf{test-free} if it does not use the \test{} command at all.
        \vfill
        \item A program is said to be a \textbf{checking program} if is of the form $M;\test(c_1,\dots,c_n)$, where $M$ is a test-free program.
    \end{itemize}
\end{frame}

\subsection{Simulating \rmfamily{\textbf{test}}}

\begin{frame}{Simulating \rmfamily{\textbf{test}} command}
    \begin{itemize}
        \item For reasons that will be clear later, we are interested in converting a general program to a checking program. This obviously is NOT possible. 
        \item So, we take an additional assumption that the sum of all counters in the general counter is bounded by some given number.
        \item We need to `simulate' the command $\test(c)$  in some way, so that we can replace all the \test s that do not occur at the end.
        
    \end{itemize}
    
\end{frame}

\begin{frame}{Simulating \rmfamily{\textbf{test}} command}
    \begin{itemize}
        \item Consider a finite set $B$ of counters. $c\in B$ and we would like to simulate $\test(c)$. For this we use \textbf{two auxiliary counters} $x,y \notin B$.
        \item The counters in $B$ are enumerated as $b_0,b_1,\dots b_{d}$ with $b_0 = c$.
    \end{itemize}
\end{frame}

\begin{frame}{Simulating \rmfamily{\textbf{test}} command}

Consider the following program:

    $$\r{simtest}_{y,B,x}(c) = \left\vert \begin{array}{l}
        \lo \\
        \quad \dec(b_1);\inc(b_0);\dec(y) \\
        \vdots \\
        \lo \\
        \quad \dec(b_d);\inc(b_{d-1});\dec(y) \\
        \lo \\
        \quad \dec(b_{d-1});\inc(b_d);\dec(y) \\
        \vdots \\
        \lo \\
        \quad \dec(b_1);\inc(b_0);\dec(y) \\
        \dec(x)^{(2)}
    \end{array} \right. $$
\end{frame}

\begin{frame}{Parsing the \r{simtest} program}
    \begin{itemize}
        \item Consider the configurations $\rho$ where
        $$y \geq Kx$$
        where $$K = \sum_{b\in B} \rho(b)$$

        \item If $y \geq Kx$ was true before the \r{simtest} program, it will be maintained even after \r{simtest}. This is because if all the \lo{}s are executed maximally $y$ decreases by at most $2K$ and $x$ decreases by $2$.
    \end{itemize}
    
\end{frame}

\begin{frame}{To be precise:}
    \begin{itemize}
        \item $\alpha \prog{\test(c);\dec(y)^{(2K)};\dec(x)^{(2)}}\beta$ with $c \in B$ and $K = \alpha(\sum_{b\in B}b)$ implies
        $$\alpha \prog{\r{simtest}_{y,B,x}(c)} \beta$$

        \item $\alpha \prog{\r{simtest}_{y,B,x}(c)} \beta$ and if $y = Kx$ is satisfied by $\beta$ then $\alpha$ satisfies the same and
        $$\alpha \prog{\test(c);\dec(y)^{(2K)};\dec(x)^{(2)}}\beta$$
        \item A program contains multiple \test s which are replaced by \r{simtest}s. If $y = Kx$ is satisfied after all the \r{simtest}s, we can say that all the \test s were satisfied.
    \end{itemize}
\end{frame}

% \begin{frame}{How exactly are we simulating \test ?}
%     \begin{itemize}
%         \item Let us say that we have a program $M = A; \test(c);A'$, where $A$ and $A'$ are test free. We can replace this by another program 
%         $$M' = $$
%     \end{itemize}
% \end{frame}

\subsection{$K$-preamplifiers}
\begin{frame}{$K$-bounded semantics}

\begin{itemize}
    \item Let $\text{Conf}_{\leq K}$ denote the set of $K$-bounded configurations, that is
    $$\text{Conf}_{\leq K} = \left\{\rho\,|\, \rho(\sum_{b\in B} b) \leq K,\, B \text{ is the set of non-zero counters}\right\}$$
    \pause
    \item $K$-bounded relation $\prog{M}_{\leq K}$ is defined as 
    $$\prog{M}_{\leq K} = \begin{cases}
        \prog{\cmd} \cap (\text{Conf}_{\leq K} \times \text{Conf}_{\leq K}) & \text{if } M = \cmd \\
        \big(\prog{M_0}\big)^*\cap (\text{Conf}_{\leq K} \times \text{Conf}_{\leq K}) & \text{if } M = \lo\\
        \prog{M_1}_{\leq K};\prog{M_2}_{\leq K} & \text{if } M = M_1;M_2\\
        \prog{M_1}_{\leq K}\cup\prog{M_2}_{\leq K} & \text{if } M = M_1 \, \Or \, M_2\\        
    \end{cases}$$
\end{itemize}
    
\end{frame}

\begin{frame}{$K$-preamplifiers}

A $K$-preamplifier for a triple $(x,y,b)$ of counters is a \textbf{checking program} $A$ such that 
\begin{itemize}
    \item $\forall$ configurations $\beta$ with $\zero \prog{A} \beta$, we have $y \geq bx$ and $\beta(c) = 0$ $\forall$ $c \notin \{x,y,b\}$.
    \item If $\beta$ satisfies $y = bx$, then $\beta(b) = K$.
    \item $\forall$ $l \geq 1$ $\exists$ $\beta$ with $\zero \prog{A} \beta$ such that $y = bx$ and $x = l$. That is, $\forall$ $l \geq 1$,
    $$\zero \prog{A} (x = l, y = Kl, b = K)$$
\end{itemize}
\pause
Can you see the use of this? Loosely speaking : We can use a preamplifier before a general program to get all the $x,y$s (related by $y = Kx$) that we might need to `fuel' \r{simtest} (as for each \r{simtest} $y$ reduces by $2K$ and $x$ reduces by $2$).
    
\end{frame}


\begin{frame}{$K$-bounded semantics}

    \begin{itemize}
        \item As mentioned earlier, we want to convert a general program to a checking program. 
        \item Obviously, there will be a trade off - we restrict the general program to $K$-bounded semantics.
        \pause
        \item Given a general program $M$ and a $K$-preamplifier $A$, we can construct a checking program $A\triangleright M$ such that 
        $$\zero \prog{M}_{\leq K} \beta $$ \textbf{iff}
        $$\zero \prog{A\triangleright M} \beta$$
    \end{itemize}
    
\end{frame}

\section{Big Picture}

 \subsection{Reduction}
 
\begin{frame}{Petri Nets and Programs}

    \begin{itemize}
        \item So far we have seen many definitions in programs, but how is this relevant to petri nets?

        \item Well, we can reduce a petri net to a program and vice versa!
    \end{itemize}
    
\end{frame}

\begin{frame}{Petri Nets to Program}

Consider a petri net $\mathcal{N} = ((P,T,F);\mu_{in})$

    \begin{itemize}
        \item For each place $p \in P$, we have a counter $c_p$. 
        \item For each transition $t \in T$ we define 
        $$M_t = \prod_{p \in {}^{\bullet} t} \dec(c_p); \prod_{p \in t^{\bullet}} \inc(c_p);$$
        \item Define 
        $$M_{in} = \prod_{p \in P}\inc(c_p)^{(\mu_{in}(p))};$$
        \item The program for petri net is 
        $$M_{\mathcal{N}} = \left\vert \begin{array}{l}
            M_{in} \\
            \lo \\
            \quad \Or _ {t \in T} M_{t} \\
    \end{array} \right.$$
    \end{itemize}
\end{frame}

\begin{frame}{Program to Petri Nets}
    \begin{itemize}
        \item First we can convert a test-free program  to a VASS.
        \item VASS to petri-nets is known.
    \end{itemize}

\end{frame}

\begin{frame}{Reachability}

    For a general program $M$, whether $\exists$ $\beta$ such that 
    $$\zero \prog{M} \beta$$
    is the reachability problem.

    \begin{itemize}
        \item For a general program, it is undecidable, even for 2-dimensions (2-counters).
        \item For checking programs, it is equivalent to reachability in petri nets.
        \item For test-free programs, it is equivalent to coverability in petri nets.
    \end{itemize}

\end{frame}

\subsection{Plan}
\begin{frame}{Rough Plan}
    \begin{itemize}
        \item Our goal was to find the complexity class of reachability problem in petri nets, which is equivalent to reachability problem in checking programs.

        \item We also know that a checking program $A\triangleright M$ can be constructed such that $\zero \prog{A\triangleright M} \beta$ iff $\zero \prog{M}_{\leq K} \beta$.

        \item Result (\textbf{R}) : For a 2-dimensional general program $M$ (2-counter program), the problem, 
        whether $\exists$ $\beta$ such that $\zero \prog{M}_{\leq K} \beta $ takes at most $\text{poly}(K)$ time, that is, it is in $\mathcal{O}(K)$ time.
    \end{itemize}
\end{frame}

\begin{frame}{Rough Plan}
    \begin{itemize}
        \item Assuming that \textbf{R} us true, we can build a checking program $A\triangleright M$, whose reachability takes $\mathcal{O}( K)$ time.
        \item This means that there exists checking programs that take $\mathcal{O}(K)$ time. Hence, reachability of checking programs is  $\mathcal{O}(K)$-hard.
        \item Now, we need to show that \textbf{R} is true and construct $A\triangleright M$.
    \end{itemize}
\end{frame}

\begin{frame}{2-counter program}
    \begin{itemize}
        \item 2-dimensional general program is Turing complete, and is equivalent to 2-counter machine.
        \item A 2-dimensional general program with $K$ bound has at most $K^{n}$ different possible states, ($n$ is the size of the program). So it should not take more than $K^n$ time to explore.
        \item We can say that reachability in $K$-bounded 2-dimensional general program takes at most $K^n$ time.
    \end{itemize}
\end{frame}



\section{Grzegorczyk Hierarchy}

\begin{frame}{Fast Growing Functions}
    Consider the following functions:
    \begin{itemize}
        \item $F_0(n) = n+1$
        \item For all $i \geq 1$,
        $F_i(n) = F^{n+1}_{i-1}(n)$
    \end{itemize}
    $F_2(n)$ is in $\textsc{Exp}$, $F_3(n)$ is in \textsc{Tower}.
\vfill
    Ackermann function $A(n)$ is defined as 
    $$A(n) = F_\omega(n)$$
    where $F_\omega(n) = F_{n+1}(n)$.
\end{frame}

\begin{frame}{Complexity Classes}
   \begin{itemize}
       \item Fast-Growing Complexity Classes:
       $$\mathbb{F}_d := \bigcup_{p \in \mathcal{F}_{< d}} \textsc{DTime}(F_d(p(n))) $$
       $p$ is a function in lower complexity classes. Example, $\textsc{Tower} = \mathbb{F}_3$.
       \item Ackermann complexity: $$\textsc{Ack} = \mathbb{F}_\omega = \bigcup_{p \in \mathcal{F}_{< \omega}} \textsc{DTime}(F_\omega(p(n)))$$

   \end{itemize}
\end{frame}

\begin{frame}{$\r{evalF}_d(v,n)$}
   \begin{itemize}
       \item For $v \in \mathbb{N}^d$,
       $$F_v(n) := F_d^{v[d]}\circ \dots \circ F_1^{v[1]}(n) $$

       \item $p$ is the minimal non-zero index and $1_{d,p}$ is unit vector in $d$-dimensions with $v[p] = 1$.
       $$\text{evalF}_d(v,n) = \begin{cases}
           (v - 1_{d,p}, 2n+1) & p = 1 \\
           (v - 1_{d,p} + (n+1)1_{d,p-1}, n) & \text{otherwise}
       \end{cases}$$
       Examples:
       \begin{enumerate}
           \item $\text{evalF}_d((1,2,4),n) = ((0,2,4),2n+1)$
           \item $\text{evalF}_d((0,0,4,0,5),n) = ((0,n+1,3,0,5),n)$
       \end{enumerate}
       Can you see what's happening?

   \end{itemize}
\end{frame}

\begin{frame}{$\r{evalF}_d(v,n)$}
   \begin{itemize}
       \item If $\text{evalF}_v(n) = (w,m)$,
       $$F_v(n) = F_w(m)$$

       \item $$\text{evalF}_d^\text{max}(v,n) = (0_d, F_v(n))$$
       In other words, $\text{evalF}_d$ applied maximal number of times.

   \end{itemize}
\end{frame}

\section{Construction}

\begin{frame}{\lo{} at most}
       Given a test-free program $M$ that does not use the counters $c,c'$, we define
       $$\loopam{c+c'}{M} = \left\vert \begin{array}{l}
            \lo \\
            \quad \dec(c);\inc(c') \\
            \lo \\
            \quad \inc(c);\dec(c');M
    \end{array} \right.$$
\end{frame}

\begin{frame}{Program for $\text{evalF}_d$}

       $$\text{evalF}_{d,p} = \left\vert \begin{array}{l}
            \dec(c_p);\inc(c_{p-1})  \\
            \lo \\
            \quad \dec(x_1,\dots,x_{d+1}); \inc(y) \\
            \loopam{x+x'}{} \\
            \quad \dec(y);\inc(x_p );\inc(x_1 ,\dots , x_{d+1} ) \\
            \loopam{c_0+c_0'}{} \\
            \quad \inc(c_{p-1}) \\
            \quad \loopam{x+x'}{} \\
            \quad \quad \dec(x') \\
            \quad \quad \loopam{b+b'}{} \\
            \quad \quad \quad \dec(y);\inc(x_p);\inc(x_p ,\dots , x_{d+1} ) \\
            \r{updateB}_{d,p} \\
            
    \end{array} \right.$$

    $$\text{evalF}_d = \Or_{p=1}^{d} \text{evalF}_{d,p}$$
\end{frame}



\begin{frame}{Program for $\text{evalF}_d$}
    For $p = 1$:
       $$\r{updateB}_{d,p} = \left\vert \begin{array}{l}
            \dec(c_0) \\
            \loopam{c_0+c_0'}{} \\
            \quad \dec(c_0');\inc(c_0) \\
            \quad \loopam{b+b'}{}\\
            \quad \quad \inc(b) \\
            \dec(c_0) 
    \end{array} \right.$$
\end{frame}

\begin{frame}{Good Configuration}
\begin{itemize}
    \item A configuration $\rho$ of counters $x,x',x_1,\dots,x_{d+1},y,b,b',c_0,c_0',c_1\dotsc_{d}$ is \textbf{good} if 
    $$x' = y = b' = c'_0 = 0$$
    $$x > 0$$
    $$b = 2^{c_0}$$
    $$x_1 = 2^{c_0}x,\, x_i = 2^{c_{i-1}}x_{i-1}$$
    \item $\rho$ encodes $(v,n) \in \mathbb{N}^d\times\mathbb{N}$
    $$\rho(c_1,\dots,c_d) = v$$
    $$\rho(c_0) = n$$
\end{itemize}
    
    
\end{frame}


\begin{frame}{Ackermannian Preamplifiers}

Following is a $K$-preamplifier for $K = 2^{F_{d+1}(n)}$
       $$\r{Ack}_{d,n} = \left\vert \begin{array}{l}
            \inc(c_0)^{(n)};\inc(c_d)^{(n+1)} \\
            \inc(x) ; \inc(x_1,\dots,x_d)^{(2^n)}; \inc(x_{d+1}^{(2^{2n+1})}) \\
            \lo \\
            \quad \inc(x) ; \inc(x_1,\dots,x_d)^{(2^n)}; \inc(x_{d+1}^{(2^{2n+1})}) \\
            \lo \\
            \quad \r{evalF}_d \\
            \lo \\
            \quad \inc(y);\dec(x_1,\dots,x_{d+1}) \\
            \lo \\
            \quad \dec(c_0) \\
            \test(x_{d+1},b',c_0',c_0,\dots,c_d) 
    \end{array} \right.$$
Also, $\text{size}(\r{Ack}_{d,n}) = \mathcal{O}(d\,4^n)$
\end{frame}

\begin{frame}{Construction of $A\triangleright M$}
       \begin{itemize}
           \item Given a $K$-preamplifier on $(x,y,b)$, $A = A';\test(c_1,\dots,c_n)$ and a program $M$,
           $$A\triangleright M = \left\vert \begin{array}{l}
            A' \\
            M' \\
            \lo \\
            \quad \dec(b) \\
            \test(x,y,b,c_1,\dots,c_n) 
    \end{array} \right.$$

     \item $M'$ is $M$ with every $\inc(c)$ replaced with $\inc(c);\dec(b)$ and every $\dec(c)$ replaced with $\dec(c);\inc(b)$.
       \end{itemize}
\end{frame}

\begin{frame}{Final Touch}
    \begin{itemize}
        \item Let $M$ be any general 2-dimensional program (at most 2 non-zero counters) with size($M$) = $n$.
        \item Does there exist a $\beta$ 
        $$\zero \prog{M}_{\leq K} \beta$$
        for $K = 2^{F_d(n)}$ is $\mathbb{F}_d$-complete.
        \pause
        \item Turing Machine (space = $\mathcal{O}(\log S)$) $\longrightarrow$ 2-counter program (space = $\mathcal{O}(S)$). Since there exists a Turing machine of space $\log S$ which is $\mathcal{O}(S)$-complete in time, we can have a $\mathcal{O(S)}$-complete 2-counter program.  
        \item NOTE that $2^{F_d(n) } \in \mathbb{F}_d$ : $$2^{F_d(n)} = \mathcal{O}(F_2(F_d(n))) \leq F_{d-1}(F_{d-1}^{n+1}(n)) = F_d(F_{d-1}(n))$$
    \end{itemize}
\end{frame}

\begin{frame}{Final Touch}
    \begin{itemize}
        \item Our whole objective is to show the lower bound $\mathbb{F}_\omega$ for reachability of a $n$-dimensional checking program. 
        \item For this we need to show one example of a program that takes time equal to $\mathcal{O}(F_\omega(n))$.
        
    \end{itemize}
\end{frame}


\begin{frame}{Final Touch}
    \begin{itemize}
        \item Its simply $A\triangleright M$ for a 2-dimensional program size($M$) = $n$. We have a ($2n+4$)-dimensional checking program with size($A\triangleright M) = \textsc{Elem}(n)$, that takes $$F_{n+1}(n) = F_\omega(n)$$ time. 
        \item Therefore, reachability of checking program is in $\mathbb{F}_\omega$. Done!
    \end{itemize}
\end{frame}



\end{document}